\chapter{\textsc{ Modélisation du fonctionnement d'un seul wagonnet $w1$ }}
\section{\textsc{Le réseau de Petri}}
	
	\begin{center}
	%\includegraphics[scale=0.5]{sim1.png}
	\captionof{figure}{\textit{Réseau de Petri du fonctionnement d'un seul wagonnet $w1$ \\}}
	\label{fig1} 
	\end{center}  

\section{\textsc{Le code C correspondant}}
	\begin{lstlisting}
		for (i=0);
		public class main 
		import package
	\end{lstlisting}

%%%%%%%%%%%%%%%%%%%%%
	
\chapter{\textsc{ Modélisation du fonctionnement de deux wagonnets $w1$ et $w2$ }}
\section{\textsc{Le réseau de Petri}}
	
	\begin{center}
	%\includegraphics[scale=0.5]{sim1.png}
	\captionof{figure}{\textit{Réseau de Petri du fonctionnement de deux wagonnets $w1$ et $w2$ \\}}
	\label{fig2} 
	\end{center}  

\section{\textsc{Le code C correspondant}}
	\begin{lstlisting}
		for (i=0);
	\end{lstlisting}


%%%%%%%%%%%%%%%%%%%%%

\chapter{\textsc{ Modélisation du fonctionnement des trois wagonnets $w1$, $w2$ et $w3$}}
\section{\textsc{Le réseau de Petri}}
	
	\begin{center}
	%\includegraphics[scale=0.5]{sim1.png}
	\captionof{figure}{\textit{Réseau de Petri du fonctionnement de deux wagonnets $w1$, $w2$ et $w3$ \\}}
	\label{fig3} 
	\end{center}  

\section{\textsc{Le code C correspondant}}
	\begin{lstlisting}
		for (i=0);
	\end{lstlisting}

%%%%%%%%%%%%%%%%%%

\chapter*{\textsc{Conclusion}}
\addcontentsline{toc}{chapter}{\textsc{Conclusion}}

	\paragraph{}