\chapter*{\textsc{Introduction}}
\addcontentsline{toc}{chapter}{\textsc{Introduction}}

	\paragraph{} Le langage JAVA inclut la possibilité de gérer l’exclusion mutuelle entre des objets ou entre les méthodes d’un objet. Plusieurs possibilités décrites en cours peuvent être utilisées pour mettre en œuvre un moniteur. La plus adaptée est celle basée sur le type d’objet « Condition » auquel sont associées les fonctions d’attente et de signal sur Condition. Le but de la manipulation est de valider les acquis du cours en créant un programme permettant de gérer l’exécution de plusieurs processus « philosophes » partageant plusieurs ressources (les baguettes...).
	

